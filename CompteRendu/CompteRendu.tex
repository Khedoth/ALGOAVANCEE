\documentclass[a4paper, titlepage]{article}

\usepackage[utf8]{inputenc} % accents
\usepackage[T1]{fontenc}      % caractères français
\usepackage{geometry}         % marges
\usepackage[francais]{babel}  % langue
\usepackage{graphicx,subfigure}         % images
\usepackage{verbatim}         % texte préformaté
\usepackage{float}




\usepackage{fancyhdr}
\pagestyle{fancy}
\usepackage{lastpage}

\renewcommand\headrulewidth{1pt} 
\fancyhead[L]{Algorithmique avancée}
\fancyhead[R]{ENSSAT}

\renewcommand\footrulewidth{1pt}
\fancyfoot[C]{\today}
\fancyfoot[L]{VYTHELINGUM - NOKAYA} 
\fancyfoot[R]{\textbf{Page \thepage/\pageref{LastPage}}}

\title{Mini-projet d'algorithmique avancée}      % renseigne le titre
\author{Kévin VYTHELINGUM, Jean-Michel NOKAYA}           %   "   "   l'auteur
\date{\today}           %   "   "   la future date de parution



\begin{document}
\maketitle
\tableofcontents
\newpage
\large

\section{Introduction}

\section{Préliminaires}

\begin{enumerate}
\item
	d max => c min
	donc somme(d_max) => somme(c_min)
	donc D => min(somme(c))
\item
	complexité = k (arbre)
	
\section{Méthodes des essais successifs}
	sol : Un candidat est un vecteur de taille n où chaque coefficient est une durée choisie parmi l'ensemble {1,..,k} (à chaque tâche on associe une durée)
	
	Si : l'ensemble des durées possibles de 1 à k
	satisfaisant(x_{i}) = $\sum_{1}^{i} x_{l} <= D$
	enregistrer(x_{i}) = T[i] <- x_{i}
			     cout <- cout + cd[i][x_{i}]
			     duree <- duree + x_{i}
			     
	Soltrouvee : i = n
	défaire(x_{i}) = T[i] <- 0
			 cout <- cout - cd[i][x_{i}]
			 duree <- duree - x_{i}
			 
	Initialisation : cout = 0; duree = 0
	
		Ces variables servent à simplifier les vérifications au niveau de satisfaisant et des conditions d'élagage.
	
\end{enumerate}

\end{document}
