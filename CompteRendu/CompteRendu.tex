\documentclass[a4paper, titlepage]{article}

\usepackage[utf8]{inputenc} % accents
\usepackage[T1]{fontenc}      % caractères français
\usepackage{geometry}         % marges
\usepackage[francais]{babel}  % langue
\usepackage{graphicx,subfigure}         % images
\usepackage{verbatim}         % texte préformaté
\usepackage{float}




\usepackage{fancyhdr}
\pagestyle{fancy}
\usepackage{lastpage}

\renewcommand\headrulewidth{1pt}
\fancyhead[L]{Algorithmique avancée}
\fancyhead[R]{ENSSAT}

\renewcommand\footrulewidth{1pt}
\fancyfoot[C]{\today}
\fancyfoot[L]{VYTHELINGUM - NOKAYA}
\fancyfoot[R]{\textbf{Page \thepage/\pageref{LastPage}}}

\title{Mini-projet d'algorithmique avancée}      % renseigne le titre
\author{Kévin VYTHELINGUM, Jean-Michel NOKAYA}           %   "   "   l'auteur
\date{\today}           %   "   "   la future date de parution



\begin{document}
\maketitle
\tableofcontents
\newpage
\large

\section{Introduction}

\section{Préliminaires}

\begin{enumerate}
\item
	\begin{eqnarray*}
	d_{max} & \Rightarrow & c_{min} \\
	\sum_{i=1}^{k} d_{max} & \Rightarrow & \sum_{i=1}^{n} c_{min} \\
	D & \Rightarrow & C_{opt}
	\end{eqnarray*}
\item
	complexité = $k^{n}$ (On peut faire un arbre pour le démontrer)

\end{enumerate}

\section{Méthodes des essais successifs}

	\subsection{Analyse}

		\paragraph{} \noindent
		Solution : un candidat est un vecteur de taille n où chaque coefficient est une durée choisie parmi l'ensemble \{1,...,k\} (à chaque tâche on associe une durée).
		On choisit d'enregistrer les choix réalisés dans un tableau T de taille n.

		\paragraph{}\noindent
		$S_{i}$ : l'ensemble des durées possibles de 1 à k

		\paragraph{}\noindent
		$satisfaisant(x_{i}) = \sum_{j=1}^{i} x_{j} \le D$

		(la somme partielle des durées choisies est inférieure à la durée maximale autorisée)

		\paragraph{}\noindent
		$enregistrer(x_{i}) = T[i] \leftarrow x_{i}$

		\paragraph{}\noindent
		\emph{soltrouvee} : $i = n$

		\paragraph{}\noindent
		$defaire(x_{i}) = T[i] \leftarrow 0$

		\paragraph{}
		Pour simplifier les vérifications au niveau de satisfaisant et des conditions d'élagage, on utilisera les variables entières \emph{cout} et \emph{duree} initialisée à 0,
		qui représenteront le cout courant et la durée courante duent à nos choix de durées.
		Elles seront mises à jour dans \emph{enregistrer} et dans \emph{défaire}.
		En effet, en notant \emph{CD} le tableau à deux dimensions ayant les coûts en ligne et les durées en colonne, on effectura dans \emph{enregistrer} :

			\begin{eqnarray*}
				\mbox{coût}  & \leftarrow & \mbox{coût} + CD[i][x_{i}] \\
				\mbox{durée} & \leftarrow & \mbox{durée} + x_{i}
			\end{eqnarray*}

		Ensuite on effectura dans \emph{défaire} :

			\begin{eqnarray*}
				\mbox{coût}  & \leftarrow & \mbox{coût} - CD[i][x_{i}] \\
				\mbox{durée} & \leftarrow & \mbox{durée} - x_{i}
			\end{eqnarray*}

	\subsection{Condition d'élagage}


	\subsection{Algorithme}

		\subsubsection{Sans élagage}

\begin{tabbing}
Procédure $ordonnancement_simple(ent i)$;
var ent k, $x_{i}$, D;
début
	$S_{i} = [1..k];$
	pour $x_{i}$ de 1 à k faire
		si $\sum_{1}^{i} x_{l} <= D$ alors
			$T[i] <- x_{i};$
			$cout <- cout + cd[i][x_{i}];$
			$duree <- duree + x_{i};$

			si i = n alors $Affiche_sol()$;
			sinon
				$ordonnancement_simple(i+1)$;
			fsi;

		T[i] <- 0;
		$cout <- cout - cd[i][x_{i}];$
		$duree <- duree - x_{i};	$

		fsi;
	fait;
fin;
\end{tabbing}

Appel : $ordonnancement_simple(1)$;

		\subsubsection{Avec élagage}

	\subsection{Complexité temporelle}

	\subsection{Expérimentations}

\section{Programmation dynamique}

	\subsection{Formule de récurrence}
	\subsection{Structure tabulaire}
	\subsection{Algorithme}
	\subsection{Complexité temporelle}
	\subsection{Complexité spatiale}

\section{Question complémentaires}
\begin{enumerate}
\item
\item
\item
\item
\item
\end{enumerate}

\section{Conclusion}


\end{document}
